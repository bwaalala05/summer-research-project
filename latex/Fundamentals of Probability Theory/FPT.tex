\documentclass{report}

% PREAMBLE
%%%%%%%%%%%%%%%%%%%%%%%%%%%%%%%%%
% PACKAGE IMPORTS
%%%%%%%%%%%%%%%%%%%%%%%%%%%%%%%%%

\usepackage{lmodern}
\usepackage[tmargin=2cm,rmargin=1in,lmargin=1in,margin=0.85in,bmargin=2cm,footskip=.2in]{geometry}
\usepackage{amsmath,amsfonts,amsthm,amssymb,mathtools}
\usepackage[varbb]{newpxmath}
\usepackage{ mathrsfs }
%\usepackage{xfrac}
\usepackage[makeroom]{cancel}
\usepackage{mathtools}
\usepackage{bookmark}
\usepackage{enumitem}
\usepackage{hyperref,theoremref}
\hypersetup{
	pdftitle={Assignment},
	colorlinks=true, linkcolor=doc!90,
	bookmarksnumbered=true,
	bookmarksopen=true
}
\usepackage[most,many,breakable]{tcolorbox}
\usepackage{xcolor}
\usepackage{varwidth}
\usepackage{varwidth}
\usepackage{etoolbox}
%\usepackage{authblk}
\usepackage{nameref}
\usepackage{multicol,array}
\usepackage{tikz-cd}
\usepackage[ruled,vlined,linesnumbered]{algorithm2e}
\usepackage{comment} % enables the use of multi-line comments (\ifx \fi) 
\usepackage{import}
\usepackage{xifthen}
\usepackage{pdfpages}
\usepackage{transparent}
\usepackage{subfiles}
\usepackage{hyperref}

\newcommand\mycommfont[1]{\footnotesize\ttfamily\textcolor{blue}{#1}}
\SetCommentSty{mycommfont}
\newcommand{\incfig}[1]{%
	\def\svgwidth{\columnwidth}
	\import{./figures/}{#1.pdf_tex}
}

\usepackage{tikzsymbols}

\definecolor{doc}{RGB}{0,0,0}

\definecolor{lemmaTitle}{RGB}{255, 166, 77}
\definecolor{lemmaBody}{RGB}{255, 218, 179}

\definecolor{defTitle}{RGB}{0, 179, 179}
\definecolor{defBody}{RGB}{230, 255, 255}

\definecolor{thmTitle}{RGB}{255, 179, 255}
\definecolor{thmBody}{RGB}{255, 230, 255}

\definecolor{crlTitle}{RGB}{255, 255, 128}
\definecolor{crlBody}{RGB}{255, 255, 204}

\definecolor{phyTitle}{RGB}{153, 153, 153}
\definecolor{phyTitle}{RGB}{179, 179, 179}

%%%%%%%%%%%%%%%%%%%%%%%%%%%%%%
% SELF MADE COLORS
%%%%%%%%%%%%%%%%%%%%%%%%%%%%%%



\definecolor{myg}{RGB}{56, 140, 70}
\definecolor{myb}{RGB}{45, 111, 177}
\definecolor{myr}{RGB}{199, 68, 64}
\definecolor{myblue}{HTML}{456B89}
\definecolor{myorange}{HTML}{D3762C}
\definecolor{mygreen}{HTML}{4a8945}
\definecolor{mypurple}{HTML}{844589}
\definecolor{mytheorembg}{HTML}{F2F2F9}
\definecolor{mytheoremfr}{HTML}{00007B}
\definecolor{mylenmabg}{HTML}{FFFAF8}
\definecolor{mylenmafr}{HTML}{983b0f}
\definecolor{mypropbg}{HTML}{f2fbfc}
\definecolor{mypropfr}{HTML}{191971}
\definecolor{myexamplebg}{HTML}{F2FBF8}
\definecolor{myexamplefr}{HTML}{88D6D1}
\definecolor{myexampleti}{HTML}{2A7F7F}
\definecolor{mydefinitbg}{HTML}{E5E5FF}
\definecolor{mydefinitfr}{HTML}{3F3FA3}
\definecolor{notesgreen}{RGB}{0,162,0}
\definecolor{myp}{RGB}{197, 92, 212}
\definecolor{mygr}{HTML}{2C3338}
\definecolor{myred}{RGB}{127,0,0}
\definecolor{myyellow}{RGB}{169,121,69}
\definecolor{myexercisebg}{HTML}{F2FBF8}
\definecolor{myexercisefg}{HTML}{88D6D1}



%%%%%%%%%%%%%%%%%%%%%%%%%%%%
% TCOLORBOX SETUPS
%%%%%%%%%%%%%%%%%%%%%%%%%%%%

\setlength{\parindent}{1cm}
%================================
% THEOREM BOX
%================================

\tcbuselibrary{theorems,skins,hooks}
\newtcbtheorem[number within=section]{Theorem}{Theorem}
{%
	enhanced,
	breakable,
	colback = mytheorembg,
	frame hidden,
	boxrule = 0sp,
	borderline west = {2pt}{0pt}{mytheoremfr},
	sharp corners,
	detach title,
	before upper = \tcbtitle\par\smallskip,
	coltitle = mytheoremfr,
	fonttitle = \bfseries\sffamily,
	description font = \mdseries,
	separator sign none,
	segmentation style={solid, mytheoremfr},
}
{th}

\tcbuselibrary{theorems,skins,hooks}
\newtcbtheorem[number within=chapter]{theorem}{Theorem}
{%
	enhanced,
	breakable,
	colback = mytheorembg,
	frame hidden,
	boxrule = 0sp,
	borderline west = {2pt}{0pt}{mytheoremfr},
	sharp corners,
	detach title,
	before upper = \tcbtitle\par\smallskip,
	coltitle = mytheoremfr,
	fonttitle = \bfseries\sffamily,
	description font = \mdseries,
	separator sign none,
	segmentation style={solid, mytheoremfr},
}
{th}


\tcbuselibrary{theorems,skins,hooks}
\newtcolorbox{Theoremcon}
{%
	enhanced
	,breakable
	,colback = mytheorembg
	,frame hidden
	,boxrule = 0sp
	,borderline west = {2pt}{0pt}{mytheoremfr}
	,sharp corners
	,description font = \mdseries
	,separator sign none
}

%================================
% Corollery
%================================
\tcbuselibrary{theorems,skins,hooks}
\newtcbtheorem[number within=section]{Corollary}{Corollary}
{%
	enhanced
	,breakable
	,colback = myp!10
	,frame hidden
	,boxrule = 0sp
	,borderline west = {2pt}{0pt}{myp!85!black}
	,sharp corners
	,detach title
	,before upper = \tcbtitle\par\smallskip
	,coltitle = myp!85!black
	,fonttitle = \bfseries\sffamily
	,description font = \mdseries
	,separator sign none
	,segmentation style={solid, myp!85!black}
}
{th}
\tcbuselibrary{theorems,skins,hooks}
\newtcbtheorem[number within=chapter]{corollary}{Corollary}
{%
	enhanced
	,breakable
	,colback = myp!10
	,frame hidden
	,boxrule = 0sp
	,borderline west = {2pt}{0pt}{myp!85!black}
	,sharp corners
	,detach title
	,before upper = \tcbtitle\par\smallskip
	,coltitle = myp!85!black
	,fonttitle = \bfseries\sffamily
	,description font = \mdseries
	,separator sign none
	,segmentation style={solid, myp!85!black}
}
{th}


%================================
% LEMMA
%================================

\tcbuselibrary{theorems,skins,hooks}
\newtcbtheorem[number within=section]{Lemma}{Lemma}
{%
	enhanced,
	breakable,
	colback = mylenmabg,
	frame hidden,
	boxrule = 0sp,
	borderline west = {2pt}{0pt}{mylenmafr},
	sharp corners,
	detach title,
	before upper = \tcbtitle\par\smallskip,
	coltitle = mylenmafr,
	fonttitle = \bfseries\sffamily,
	description font = \mdseries,
	separator sign none,
	segmentation style={solid, mylenmafr},
}
{th}

\tcbuselibrary{theorems,skins,hooks}
\newtcbtheorem[number within=chapter]{lenma}{Lenma}
{%
	enhanced,
	breakable,
	colback = mylenmabg,
	frame hidden,
	boxrule = 0sp,
	borderline west = {2pt}{0pt}{mylenmafr},
	sharp corners,
	detach title,
	before upper = \tcbtitle\par\smallskip,
	coltitle = mylenmafr,
	fonttitle = \bfseries\sffamily,
	description font = \mdseries,
	separator sign none,
	segmentation style={solid, mylenmafr},
}
{th}

%================================
% Exercise
%================================

\tcbuselibrary{theorems,skins,hooks}
\newtcbtheorem[number within=section]{Exercise}{Exercise}
{%
	enhanced,
	breakable,
	colback = myexercisebg,
	frame hidden,
	boxrule = 0sp,
	borderline west = {2pt}{0pt}{myexercisefg},
	sharp corners,
	detach title,
	before upper = \tcbtitle\par\smallskip,
	coltitle = myexercisefg,
	fonttitle = \bfseries\sffamily,
	description font = \mdseries,
	separator sign none,
	segmentation style={solid, myexercisefg},
}
{th}

\tcbuselibrary{theorems,skins,hooks}
\newtcbtheorem[number within=chapter]{exercise}{Exercise}
{%
	enhanced,
	breakable,
	colback = myexercisebg,
	frame hidden,
	boxrule = 0sp,
	borderline west = {2pt}{0pt}{myexercisefg},
	sharp corners,
	detach title,
	before upper = \tcbtitle\par\smallskip,
	coltitle = myexercisefg,
	fonttitle = \bfseries\sffamily,
	description font = \mdseries,
	separator sign none,
	segmentation style={solid, myexercisefg},
}
{th}



%================================
% PROPOSITION
%================================

\tcbuselibrary{theorems,skins,hooks}
\newtcbtheorem[number within=section]{Prop}{Proposition}
{%
	enhanced,
	breakable,
	colback = mypropbg,
	frame hidden,
	boxrule = 0sp,
	borderline west = {2pt}{0pt}{mypropfr},
	sharp corners,
	detach title,
	before upper = \tcbtitle\par\smallskip,
	coltitle = mypropfr,
	fonttitle = \bfseries\sffamily,
	description font = \mdseries,
	separator sign none,
	segmentation style={solid, mypropfr},
}
{th}

\tcbuselibrary{theorems,skins,hooks}
\newtcbtheorem[number within=chapter]{prop}{Proposition}
{%
	enhanced,
	breakable,
	colback = mypropbg,
	frame hidden,
	boxrule = 0sp,
	borderline west = {2pt}{0pt}{mypropfr},
	sharp corners,
	detach title,
	before upper = \tcbtitle\par\smallskip,
	coltitle = mypropfr,
	fonttitle = \bfseries\sffamily,
	description font = \mdseries,
	separator sign none,
	segmentation style={solid, mypropfr},
}
{th}


%================================
% EXAMPLE BOX
%================================

\newtcbtheorem[number within=section]{Example}{Example}
{%
	colback = myexamplebg
	,breakable
	,colframe = myexamplefr
	,coltitle = myexampleti
	,boxrule = 1pt
	,sharp corners
	,detach title
	,before upper=\tcbtitle\par\smallskip
	,fonttitle = \bfseries
	,description font = \mdseries
	,separator sign none
	,description delimiters parenthesis
}
{ex}

\newtcbtheorem[number within=chapter]{example}{Example}
{%
	colback = myexamplebg
	,breakable
	,colframe = myexamplefr
	,coltitle = myexampleti
	,boxrule = 1pt
	,sharp corners
	,detach title
	,before upper=\tcbtitle\par\smallskip
	,fonttitle = \bfseries
	,description font = \mdseries
	,separator sign none
	,description delimiters parenthesis
}
{ex}

%================================
% DEFINITION BOX
%================================

\newtcbtheorem[number within=section]{Definition}{Definition}{enhanced,
	before skip=2mm,after skip=2mm, colback=red!5,colframe=red!80!black,boxrule=0.5mm,
	attach boxed title to top left={xshift=1cm,yshift*=1mm-\tcboxedtitleheight}, varwidth boxed title*=-3cm,
	boxed title style={frame code={
					\path[fill=tcbcolback]
					([yshift=-1mm,xshift=-1mm]frame.north west)
					arc[start angle=0,end angle=180,radius=1mm]
					([yshift=-1mm,xshift=1mm]frame.north east)
					arc[start angle=180,end angle=0,radius=1mm];
					\path[left color=tcbcolback!60!black,right color=tcbcolback!60!black,
						middle color=tcbcolback!80!black]
					([xshift=-2mm]frame.north west) -- ([xshift=2mm]frame.north east)
					[rounded corners=1mm]-- ([xshift=1mm,yshift=-1mm]frame.north east)
					-- (frame.south east) -- (frame.south west)
					-- ([xshift=-1mm,yshift=-1mm]frame.north west)
					[sharp corners]-- cycle;
				},interior engine=empty,
		},
	fonttitle=\bfseries,
	title={#2},#1}{def}
\newtcbtheorem[number within=chapter]{definition}{Definition}{enhanced,
	before skip=2mm,after skip=2mm, colback=red!5,colframe=red!80!black,boxrule=0.5mm,
	attach boxed title to top left={xshift=1cm,yshift*=1mm-\tcboxedtitleheight}, varwidth boxed title*=-3cm,
	boxed title style={frame code={
					\path[fill=tcbcolback]
					([yshift=-1mm,xshift=-1mm]frame.north west)
					arc[start angle=0,end angle=180,radius=1mm]
					([yshift=-1mm,xshift=1mm]frame.north east)
					arc[start angle=180,end angle=0,radius=1mm];
					\path[left color=tcbcolback!60!black,right color=tcbcolback!60!black,
						middle color=tcbcolback!80!black]
					([xshift=-2mm]frame.north west) -- ([xshift=2mm]frame.north east)
					[rounded corners=1mm]-- ([xshift=1mm,yshift=-1mm]frame.north east)
					-- (frame.south east) -- (frame.south west)
					-- ([xshift=-1mm,yshift=-1mm]frame.north west)
					[sharp corners]-- cycle;
				},interior engine=empty,
		},
	fonttitle=\bfseries,
	title={#2},#1}{def}



%================================
% EXERCISE BOX
%================================

\makeatletter
\newtcbtheorem{question}{Question}{enhanced,
	breakable,
	colback=white,
	colframe=myb!80!black,
	attach boxed title to top left={yshift*=-\tcboxedtitleheight},
	fonttitle=\bfseries,
	title={#2},
	boxed title size=title,
	boxed title style={%
			sharp corners,
			rounded corners=northwest,
			colback=tcbcolframe,
			boxrule=0pt,
		},
	underlay boxed title={%
			\path[fill=tcbcolframe] (title.south west)--(title.south east)
			to[out=0, in=180] ([xshift=5mm]title.east)--
			(title.center-|frame.east)
			[rounded corners=\kvtcb@arc] |-
			(frame.north) -| cycle;
		},
	#1
}{def}
\makeatother

%================================
% SOLUTION BOX
%================================

\makeatletter
\newtcolorbox{solution}{enhanced,
	breakable,
	colback=white,
	colframe=myg!80!black,
	attach boxed title to top left={yshift*=-\tcboxedtitleheight},
	title=Solution,
	boxed title size=title,
	boxed title style={%
			sharp corners,
			rounded corners=northwest,
			colback=tcbcolframe,
			boxrule=0pt,
		},
	underlay boxed title={%
			\path[fill=tcbcolframe] (title.south west)--(title.south east)
			to[out=0, in=180] ([xshift=5mm]title.east)--
			(title.center-|frame.east)
			[rounded corners=\kvtcb@arc] |-
			(frame.north) -| cycle;
		},
}
\makeatother

%================================
% Question BOX
%================================

\makeatletter
\newtcbtheorem{qstion}{Question}{enhanced,
	breakable,
	colback=white,
	colframe=mygr,
	attach boxed title to top left={yshift*=-\tcboxedtitleheight},
	fonttitle=\bfseries,
	title={#2},
	boxed title size=title,
	boxed title style={%
			sharp corners,
			rounded corners=northwest,
			colback=tcbcolframe,
			boxrule=0pt,
		},
	underlay boxed title={%
			\path[fill=tcbcolframe] (title.south west)--(title.south east)
			to[out=0, in=180] ([xshift=5mm]title.east)--
			(title.center-|frame.east)
			[rounded corners=\kvtcb@arc] |-
			(frame.north) -| cycle;
		},
	#1
}{def}
\makeatother

\newtcbtheorem[number within=chapter]{wconc}{Wrong Concept}{
	breakable,
	enhanced,
	colback=white,
	colframe=myr,
	arc=0pt,
	outer arc=0pt,
	fonttitle=\bfseries\sffamily\large,
	colbacktitle=myr,
	attach boxed title to top left={},
	boxed title style={
			enhanced,
			skin=enhancedfirst jigsaw,
			arc=3pt,
			bottom=0pt,
			interior style={fill=myr}
		},
	#1
}{def}



%================================
% NOTE BOX
%================================

\usetikzlibrary{arrows,calc,shadows.blur}
\tcbuselibrary{skins}
\newtcolorbox{note}[1][]{%
	enhanced jigsaw,
	colback=gray!20!white,%
	colframe=gray!80!black,
	size=small,
	boxrule=1pt,
	title=\textbf{Note:-},
	halign title=flush center,
	coltitle=black,
	breakable,
	drop shadow=black!50!white,
	attach boxed title to top left={xshift=1cm,yshift=-\tcboxedtitleheight/2,yshifttext=-\tcboxedtitleheight/2},
	minipage boxed title=1.5cm,
	boxed title style={%
			colback=white,
			size=fbox,
			boxrule=1pt,
			boxsep=2pt,
			underlay={%
					\coordinate (dotA) at ($(interior.west) + (-0.5pt,0)$);
					\coordinate (dotB) at ($(interior.east) + (0.5pt,0)$);
					\begin{scope}
						\clip (interior.north west) rectangle ([xshift=3ex]interior.east);
						\filldraw [white, blur shadow={shadow opacity=60, shadow yshift=-.75ex}, rounded corners=2pt] (interior.north west) rectangle (interior.south east);
					\end{scope}
					\begin{scope}[gray!80!black]
						\fill (dotA) circle (2pt);
						\fill (dotB) circle (2pt);
					\end{scope}
				},
		},
	#1,
}

%%%%%%%%%%%%%%%%%%%%%%%%%%%%%%
% SELF MADE COMMANDS
%%%%%%%%%%%%%%%%%%%%%%%%%%%%%%


\newcommand{\thm}[2]{\begin{Theorem}{#1}{}#2\end{Theorem}}
\newcommand{\cor}[2]{\begin{Corollary}{#1}{}#2\end{Corollary}}
\newcommand{\mlenma}[2]{\begin{Lemma}{#1}{}#2\end{Lemma}}
\newcommand{\mer}[2]{\begin{Exercise}{#1}{}#2\end{Exercise}}
\newcommand{\mprop}[2]{\begin{Prop}{#1}{}#2\end{Prop}}
\newcommand{\clm}[3]{\begin{claim}{#1}{#2}#3\end{claim}}
\newcommand{\wc}[2]{\begin{wconc}{#1}{}\setlength{\parindent}{1cm}#2\end{wconc}}
\newcommand{\thmcon}[1]{\begin{Theoremcon}{#1}\end{Theoremcon}}
\newcommand{\ex}[2]{\begin{Example}{#1}{}#2\end{Example}}
\newcommand{\dfn}[2]{\begin{Definition}[colbacktitle=red!75!black]{#1}{}#2\end{Definition}}
\newcommand{\dfnc}[2]{\begin{definition}[colbacktitle=red!75!black]{#1}{}#2\end{definition}}
\newcommand{\qs}[2]{\begin{question}{#1}{}#2\end{question}}
\newcommand{\pf}[2]{\begin{myproof}[#1]#2\end{myproof}}
\newcommand{\nt}[1]{\begin{note}#1\end{note}}

\newcommand*\circled[1]{\tikz[baseline=(char.base)]{
		\node[shape=circle,draw,inner sep=1pt] (char) {#1};}}
\newcommand\getcurrentref[1]{%
	\ifnumequal{\value{#1}}{0}
	{??}
	{\the\value{#1}}%
}
\newcommand{\getCurrentSectionNumber}{\getcurrentref{section}}
\newenvironment{myproof}[1][\proofname]{%
	\proof[\bfseries #1: ]%
}{\endproof}



%%%%%%%%%%%%%%%%%%%%%%%%%%%%%%%%%%%%%%%%%%%
% TABLE OF CONTENTS
%%%%%%%%%%%%%%%%%%%%%%%%%%%%%%%%%%%%%%%%%%%

\usepackage{tikz}
\definecolor{doc}{RGB}{0,60,110}
\usepackage{titletoc}
\contentsmargin{0cm}
\titlecontents{chapter}[3.7pc]
{\addvspace{30pt}%
	\begin{tikzpicture}[remember picture, overlay]%
		\draw[fill=doc!60,draw=doc!60] (-7,-.1) rectangle (-0.9,.5);%
		\pgftext[left,x=-3.7cm,y=0.2cm]{\color{white}\Large\sc\bfseries Chapter\ \thecontentslabel};%
	\end{tikzpicture}\color{doc!60}\large\sc\bfseries}%
{}
{}
{\;\titlerule\;\large\sc\bfseries Page \thecontentspage
	\begin{tikzpicture}[remember picture, overlay]
		\draw[fill=doc!60,draw=doc!60] (2pt,0) rectangle (4,0.1pt);
	\end{tikzpicture}}%
\titlecontents{section}[3.7pc]
{\addvspace{2pt}}
{\contentslabel[\thecontentslabel]{2pc}}
{}
{\hfill\small \thecontentspage}
[]
\titlecontents*{subsection}[3.7pc]
{\addvspace{-1pt}\small}
{}
{}
{\ --- \small\thecontentspage}
[ \textbullet\ ][]

\makeatletter
\renewcommand{\tableofcontents}{%
	\chapter*{%
	  \vspace*{-20\p@}%
	  \begin{tikzpicture}[remember picture, overlay]%
		  \pgftext[right,x=15cm,y=0.2cm]{\color{doc!60}\Huge\sc\bfseries \contentsname};%
		  \draw[fill=doc!60,draw=doc!60] (13,-.75) rectangle (20,1);%
		  \clip (13,-.75) rectangle (20,1);
		  \pgftext[right,x=15cm,y=0.2cm]{\color{white}\Huge\sc\bfseries \contentsname};%
	  \end{tikzpicture}}%
	\@starttoc{toc}}
\makeatother

\makeatletter
\renewcommand*\env@matrix[1][*\c@MaxMatrixCols c]{%
	\hskip -\arraycolsep
	\let\@ifnextchar\new@ifnextchar
	\array{#1}}
\makeatother
\newcommand{\eps}{\epsilon}
\newcommand{\veps}{\varepsilon}
\newcommand{\Qed}{\begin{flushright}\qed\end{flushright}}

\newcommand{\parinn}{\setlength{\parindent}{1cm}}
\newcommand{\parinf}{\setlength{\parindent}{0cm}}

% \newcommand{\norm}{\|\cdot\|}
\newcommand{\inorm}{\norm_{\infty}}
\newcommand{\opensets}{\{V_{\alpha}\}_{\alpha\in I}}
\newcommand{\oset}{V_{\alpha}}
\newcommand{\opset}[1]{V_{\alpha_{#1}}}
\newcommand{\lub}{\text{lub}}
\newcommand{\del}[2]{\frac{\partial #1}{\partial #2}}
\newcommand{\Del}[3]{\frac{\partial^{#1} #2}{\partial^{#1} #3}}
\newcommand{\deld}[2]{\dfrac{\partial #1}{\partial #2}}
\newcommand{\Deld}[3]{\dfrac{\partial^{#1} #2}{\partial^{#1} #3}}
\newcommand{\der}[2]{\frac{\mathrm{d} #1}{\mathrm{d} #2}}
% \newcommand{\ddd}[3]{\frac{\mathrm{d}^{#3} #1}{\mathrm{d}^{#3} #2}}
\newcommand{\lm}{\lambda}
\newcommand{\uin}{\mathbin{\rotatebox[origin=c]{90}{$\in$}}}
\newcommand{\usubset}{\mathbin{\rotatebox[origin=c]{90}{$\subset$}}}
\newcommand{\lt}{\left}
\newcommand{\rt}{\right}
\newcommand{\bs}[1]{\boldsymbol{#1}}
\newcommand{\exs}{\exists}
\newcommand{\st}{\strut}
\newcommand{\dps}[1]{\displaystyle{#1}}
\newcommand{\id}{\text{id}}


\newcommand{\sol}{\setlength{\parindent}{0cm}\textbf{\textit{Solution:}}\setlength{\parindent}{1cm} }
\newcommand{\solve}[1]{\setlength{\parindent}{0cm}\textbf{\textit{Solution: }}\setlength{\parindent}{1cm}#1 \Qed}
\newcommand{\lsp}{\langle}
\newcommand{\rsp}{\rangle}
\newcommand{\bzero}{\textbf{0}}
\newcommand{\Rn}{\RR^n}
\newcommand{\sbst}{\subset}

\newenvironment{amatrix}[1]{%
	\left(\begin{array}{@{}*{#1}{c}|c@{}}
	}{%
	\end{array}\right)
}
% number sets
\newcommand{\RR}[1][]{\ensuremath{\ifstrempty{#1}{\mathbb{R}}{\mathbb{R}^{#1}}}}
\newcommand{\NN}[1][]{\ensuremath{\ifstrempty{#1}{\mathbb{N}}{\mathbb{N}^{#1}}}}
\newcommand{\ZZ}[1][]{\ensuremath{\ifstrempty{#1}{\mathbb{Z}}{\mathbb{Z}^{#1}}}}
\newcommand{\QQ}[1][]{\ensuremath{\ifstrempty{#1}{\mathbb{Q}}{\mathbb{Q}^{#1}}}}
\newcommand{\CC}[1][]{\ensuremath{\ifstrempty{#1}{\mathbb{C}}{\mathbb{C}^{#1}}}}
\newcommand{\PP}[1][]{\ensuremath{\ifstrempty{#1}{\mathbb{P}}{\mathbb{P}^{#1}}}}
\newcommand{\HH}[1][]{\ensuremath{\ifstrempty{#1}{\mathbb{H}}{\mathbb{H}^{#1}}}}
\newcommand{\FF}[1][]{\ensuremath{\ifstrempty{#1}{\mathbb{F}}{\mathbb{F}^{#1}}}}
% expected value
\newcommand{\EE}{\ensuremath{\mathbb{E}}}

%---------------------------------------
% BlackBoard Math Fonts :-
%---------------------------------------

%Captital Letters
\newcommand{\bbA}{\mathbb{A}}	\newcommand{\bbB}{\mathbb{B}}
\newcommand{\bbC}{\mathbb{C}}	\newcommand{\bbD}{\mathbb{D}}
\newcommand{\bbE}{\mathbb{E}}	\newcommand{\bbF}{\mathbb{F}}
\newcommand{\bbG}{\mathbb{G}}	\newcommand{\bbH}{\mathbb{H}}
\newcommand{\bbI}{\mathbb{I}}	\newcommand{\bbJ}{\mathbb{J}}
\newcommand{\bbK}{\mathbb{K}}	\newcommand{\bbL}{\mathbb{L}}
\newcommand{\bbM}{\mathbb{M}}	\newcommand{\bbN}{\mathbb{N}}
\newcommand{\bbO}{\mathbb{O}}	\newcommand{\bbP}{\mathbb{P}}
\newcommand{\bbQ}{\mathbb{Q}}	\newcommand{\bbR}{\mathbb{R}}
\newcommand{\bbS}{\mathbb{S}}	\newcommand{\bbT}{\mathbb{T}}
\newcommand{\bbU}{\mathbb{U}}	\newcommand{\bbV}{\mathbb{V}}
\newcommand{\bbW}{\mathbb{W}}	\newcommand{\bbX}{\mathbb{X}}
\newcommand{\bbY}{\mathbb{Y}}	\newcommand{\bbZ}{\mathbb{Z}}

%---------------------------------------
% MathCal Fonts :-
%---------------------------------------

%Captital Letters
\newcommand{\mcA}{\mathcal{A}}	\newcommand{\mcB}{\mathcal{B}}
\newcommand{\mcC}{\mathcal{C}}	\newcommand{\mcD}{\mathcal{D}}
\newcommand{\mcE}{\mathcal{E}}	\newcommand{\mcF}{\mathcal{F}}
\newcommand{\mcG}{\mathcal{G}}	\newcommand{\mcH}{\mathcal{H}}
\newcommand{\mcI}{\mathcal{I}}	\newcommand{\mcJ}{\mathcal{J}}
\newcommand{\mcK}{\mathcal{K}}	\newcommand{\mcL}{\mathcal{L}}
\newcommand{\mcM}{\mathcal{M}}	\newcommand{\mcN}{\mathcal{N}}
\newcommand{\mcO}{\mathcal{O}}	\newcommand{\mcP}{\mathcal{P}}
\newcommand{\mcQ}{\mathcal{Q}}	\newcommand{\mcR}{\mathcal{R}}
\newcommand{\mcS}{\mathcal{S}}	\newcommand{\mcT}{\mathcal{T}}
\newcommand{\mcU}{\mathcal{U}}	\newcommand{\mcV}{\mathcal{V}}
\newcommand{\mcW}{\mathcal{W}}	\newcommand{\mcX}{\mathcal{X}}
\newcommand{\mcY}{\mathcal{Y}}	\newcommand{\mcZ}{\mathcal{Z}}



%---------------------------------------
% Bold Math Fonts :-
%---------------------------------------

%Captital Letters
\newcommand{\bmA}{\boldsymbol{A}}	\newcommand{\bmB}{\boldsymbol{B}}
\newcommand{\bmC}{\boldsymbol{C}}	\newcommand{\bmD}{\boldsymbol{D}}
\newcommand{\bmE}{\boldsymbol{E}}	\newcommand{\bmF}{\boldsymbol{F}}
\newcommand{\bmG}{\boldsymbol{G}}	\newcommand{\bmH}{\boldsymbol{H}}
\newcommand{\bmI}{\boldsymbol{I}}	\newcommand{\bmJ}{\boldsymbol{J}}
\newcommand{\bmK}{\boldsymbol{K}}	\newcommand{\bmL}{\boldsymbol{L}}
\newcommand{\bmM}{\boldsymbol{M}}	\newcommand{\bmN}{\boldsymbol{N}}
\newcommand{\bmO}{\boldsymbol{O}}	\newcommand{\bmP}{\boldsymbol{P}}
\newcommand{\bmQ}{\boldsymbol{Q}}	\newcommand{\bmR}{\boldsymbol{R}}
\newcommand{\bmS}{\boldsymbol{S}}	\newcommand{\bmT}{\boldsymbol{T}}
\newcommand{\bmU}{\boldsymbol{U}}	\newcommand{\bmV}{\boldsymbol{V}}
\newcommand{\bmW}{\boldsymbol{W}}	\newcommand{\bmX}{\boldsymbol{X}}
\newcommand{\bmY}{\boldsymbol{Y}}	\newcommand{\bmZ}{\boldsymbol{Z}}
%Small Letters
\newcommand{\bma}{\boldsymbol{a}}	\newcommand{\bmb}{\boldsymbol{b}}
\newcommand{\bmc}{\boldsymbol{c}}	\newcommand{\bmd}{\boldsymbol{d}}
\newcommand{\bme}{\boldsymbol{e}}	\newcommand{\bmf}{\boldsymbol{f}}
\newcommand{\bmg}{\boldsymbol{g}}	\newcommand{\bmh}{\boldsymbol{h}}
\newcommand{\bmi}{\boldsymbol{i}}	\newcommand{\bmj}{\boldsymbol{j}}
\newcommand{\bmk}{\boldsymbol{k}}	\newcommand{\bml}{\boldsymbol{l}}
\newcommand{\bmm}{\boldsymbol{m}}	\newcommand{\bmn}{\boldsymbol{n}}
\newcommand{\bmo}{\boldsymbol{o}}	\newcommand{\bmp}{\boldsymbol{p}}
\newcommand{\bmq}{\boldsymbol{q}}	\newcommand{\bmr}{\boldsymbol{r}}
\newcommand{\bms}{\boldsymbol{s}}	\newcommand{\bmt}{\boldsymbol{t}}
\newcommand{\bmu}{\boldsymbol{u}}	\newcommand{\bmv}{\boldsymbol{v}}
\newcommand{\bmw}{\boldsymbol{w}}	\newcommand{\bmx}{\boldsymbol{x}}
\newcommand{\bmy}{\boldsymbol{y}}	\newcommand{\bmz}{\boldsymbol{z}}

%---------------------------------------
% Scr Math Fonts :-
%---------------------------------------

\newcommand{\sA}{{\mathscr{A}}}   \newcommand{\sB}{{\mathscr{B}}}
\newcommand{\sC}{{\mathscr{C}}}   \newcommand{\sD}{{\mathscr{D}}}
\newcommand{\sE}{{\mathscr{E}}}   \newcommand{\sF}{{\mathscr{F}}}
\newcommand{\sG}{{\mathscr{G}}}   \newcommand{\sH}{{\mathscr{H}}}
\newcommand{\sI}{{\mathscr{I}}}   \newcommand{\sJ}{{\mathscr{J}}}
\newcommand{\sK}{{\mathscr{K}}}   \newcommand{\sL}{{\mathscr{L}}}
\newcommand{\sM}{{\mathscr{M}}}   \newcommand{\sN}{{\mathscr{N}}}
\newcommand{\sO}{{\mathscr{O}}}   \newcommand{\sP}{{\mathscr{P}}}
\newcommand{\sQ}{{\mathscr{Q}}}   \newcommand{\sR}{{\mathscr{R}}}
\newcommand{\sS}{{\mathscr{S}}}   \newcommand{\sT}{{\mathscr{T}}}
\newcommand{\sU}{{\mathscr{U}}}   \newcommand{\sV}{{\mathscr{V}}}
\newcommand{\sW}{{\mathscr{W}}}   \newcommand{\sX}{{\mathscr{X}}}
\newcommand{\sY}{{\mathscr{Y}}}   \newcommand{\sZ}{{\mathscr{Z}}}


%---------------------------------------
% Math Fraktur Font
%---------------------------------------

%Captital Letters
\newcommand{\mfA}{\mathfrak{A}}	\newcommand{\mfB}{\mathfrak{B}}
\newcommand{\mfC}{\mathfrak{C}}	\newcommand{\mfD}{\mathfrak{D}}
\newcommand{\mfE}{\mathfrak{E}}	\newcommand{\mfF}{\mathfrak{F}}
\newcommand{\mfG}{\mathfrak{G}}	\newcommand{\mfH}{\mathfrak{H}}
\newcommand{\mfI}{\mathfrak{I}}	\newcommand{\mfJ}{\mathfrak{J}}
\newcommand{\mfK}{\mathfrak{K}}	\newcommand{\mfL}{\mathfrak{L}}
\newcommand{\mfM}{\mathfrak{M}}	\newcommand{\mfN}{\mathfrak{N}}
\newcommand{\mfO}{\mathfrak{O}}	\newcommand{\mfP}{\mathfrak{P}}
\newcommand{\mfQ}{\mathfrak{Q}}	\newcommand{\mfR}{\mathfrak{R}}
\newcommand{\mfS}{\mathfrak{S}}	\newcommand{\mfT}{\mathfrak{T}}
\newcommand{\mfU}{\mathfrak{U}}	\newcommand{\mfV}{\mathfrak{V}}
\newcommand{\mfW}{\mathfrak{W}}	\newcommand{\mfX}{\mathfrak{X}}
\newcommand{\mfY}{\mathfrak{Y}}	\newcommand{\mfZ}{\mathfrak{Z}}
%Small Letters
\newcommand{\mfa}{\mathfrak{a}}	\newcommand{\mfb}{\mathfrak{b}}
\newcommand{\mfc}{\mathfrak{c}}	\newcommand{\mfd}{\mathfrak{d}}
\newcommand{\mfe}{\mathfrak{e}}	\newcommand{\mff}{\mathfrak{f}}
\newcommand{\mfg}{\mathfrak{g}}	\newcommand{\mfh}{\mathfrak{h}}
\newcommand{\mfi}{\mathfrak{i}}	\newcommand{\mfj}{\mathfrak{j}}
\newcommand{\mfk}{\mathfrak{k}}	\newcommand{\mfl}{\mathfrak{l}}
\newcommand{\mfm}{\mathfrak{m}}	\newcommand{\mfn}{\mathfrak{n}}
\newcommand{\mfo}{\mathfrak{o}}	\newcommand{\mfp}{\mathfrak{p}}
\newcommand{\mfq}{\mathfrak{q}}	\newcommand{\mfr}{\mathfrak{r}}
\newcommand{\mfs}{\mathfrak{s}}	\newcommand{\mft}{\mathfrak{t}}
\newcommand{\mfu}{\mathfrak{u}}	\newcommand{\mfv}{\mathfrak{v}}
\newcommand{\mfw}{\mathfrak{w}}	\newcommand{\mfx}{\mathfrak{x}}
\newcommand{\mfy}{\mathfrak{y}}	\newcommand{\mfz}{\mathfrak{z}}

% GREEK LETTERS
\newcommand{\ga}{\alpha}
\newcommand{\gb}{\beta}
\newcommand{\gc}{\gamma}
\newcommand{\gd}{\delta}

\newcommand{\diff}{\setminus}
\setlength{\parindent}{0pt}
\renewcommand{\baselinestretch}{1.2}
\usepackage{mathtools}

% SHORTCUTS
\newcommand{\var}{\sigma^2}
\newcommand{\sm}{\bar{x}}
\newcommand{\estm}{\hat{\gb_1}}
\newcommand{\estc}{\hat{\gb_0}}
\newcommand{\Om}{\Omega}
\newcommand{\om}{\omega}
\newcommand{\pt}{\partial}
\newcommand{\Rk}{\RR^K}
\newcommand{\RX}{R_{\bmX}}
\newcommand{\Rx}{R_X}
\newcommand\myeq{\stackrel{\mathclap{\normalfont\mbox{D}}}{=}}
\DeclareMathOperator{\cov}{cov}
\DeclareMathOperator{\Var}{Var}
\DeclareMathOperator{\bin}{Bin}
\DeclareMathOperator{\po}{Po}
\DeclareMathOperator{\vect}{vec}


\title{\LARGE{Fundamentals of Probability Theory\\ - Notes -}}
\author{Lim Zi Xiang}

\begin{document}
	\maketitle
	\pdfbookmark[section]{\contentsname}{toc}
	\tableofcontents
	%\pagebreak
	
	%%%%%%%%%%%%%%%%%%%%%%%%%%%%%%%%%%%%%%%%%%%%%%%%%%%%%%
	%$% CHAPTER 1: Random variables and random vectors $$$
	%%%%%%%%%%%%%%%%%%%%%%%%%%%%%%%%%%%%%%%%%%%%%%%%%%%%%%
	\chapter{Random variables and random vectors}
	
	%%% 1.1 | Random vectors $$$
	\section{Random vectors}
	\dfn{Random vector}{
	Let $\Omega$ be a sample space. A \textbf{random vector} $\bmX$ is a function from the sample space $\Omega$ to the set of $K$-dimensional real vectors $\RR^K$:
	$$\bmX:\Omega\rightarrow\RR^K.$$
	}
	\vspace{2mm}
	To put it simply, a random vector is a vector whose value depends on the outcome of the probabilistic experiment. The real vector $\bmX(\omega)$ associated to a sample point $\om\in\Om$ is a \textbf{realisation} of the random vector. The set of all possible realisations is the \textbf{support}, denoted $R_X$.
	
	\nt{
	We denote the probability of an event $E\subseteq\Omega$ by $P(E)$. We use the following conventions when dealing with random vectors:
	\begin{itemize}
		\item For $A\subseteq\RR^K$, $P(\bmX\in A)=P(\{\omega\in\Omega:\bmX(\omega)\in A\})$.
		\item For $A\subseteq\RR^K$, $P_{\bmX}(A)=P(\bmX\in A)$. \\
		It is very common in applied work to build statistical models where a random vector $X$ is defined by directly specifying $P_X$ and omitting the specification of the sample space $\Omega$.
		\item We often write $\bmX$ to mean $\bmX(\omega)$.\\
	\end{itemize}
	}
	\vspace{1mm}
	\ex{Defining a random vector on a sample space}{
	Two coins are toseed. The possible outcomes of each toss can be either tail (T) or head (H). The sample space is
	$$\Omega=\{TT,TH,HT,HH\}.$$
	
	The four possible outcomes are assigned equal probabilities:
	$$P(\{TT\})=P(\{TH\})=P(\{HT\})=P(\{HH\})=\frac{1}{4}.$$
	If the outcome is tails, we win a dollar, otherwise we lose one dollar. A 2D random vector $\bmX$ indicates the amount we win on each toss:
	
	$$\bmX(\om)=\begin{cases}
		\begin{pmatrix}
			1 & 1
		\end{pmatrix} & \text{ if }\om=TT\\
		\begin{pmatrix}
			1 & -1
		\end{pmatrix} & \text{ if }\om=TH\\
		\begin{pmatrix}
			-1 & 1
		\end{pmatrix} & \text{ if }\om=HT\\
		\begin{pmatrix}
			-1 & -1
		\end{pmatrix} & \text{ if }\om=HH\\
	\end{cases}$$ 
	The probability of winning one dollar on both tosses is 
	$$P\lt(\bmX=\begin{pmatrix}
		1 & 1
	\end{pmatrix}\rt)=P\lt(\{\om\in\Om:\bmX(\om)=\begin{pmatrix}
	1 & 1
	\end{pmatrix}\}\rt)=P(\{TT\})=\frac{1}{4}.$$
	The probability of losing one dollar on the second toss is
	$$P(X_2=-1)=P(\{\om\in\Om:X_2(\om)=-1\})=P(\{TH,HH\})=\frac{1}{2}.$$
	}
	\subsection{Discrete random vectors} % 1.1.1
	\dfn{Discrete random vector}{
	A random vector $\bmX$ is \textbf{discrete} iif
	\begin{enumerate}
		\item its support $R_{\bmX}$ is a countable set;
		\item there is a function $p_{\bmX}:\RR^K\rightarrow[0,1]$, called the \textbf{joint probability mass function} of $\bmX$, such that for any $\bmx\in\RR^K$:
		$$p_{\bmX}(\bmx)=\begin{cases}
			P(\bmX=\bmx) & \text{ if } \bmx\in R_{\bmX};\\
			0 & \text{ if }\bmx\notin R_{\bmX}.
		\end{cases}$$
	\end{enumerate}
	}
	\nt{
	The following are equivalent notations used interchangeably to indicate the joint pmf:
	$$p_{\bmX}(x)=p_{\bmX}(x_1,\ldots,x_K)=p_{X_1,\ldots,X_k}(x_1,\ldots,x_K).$$
	}
	\vspace{1mm}
	\ex{}{
	Suppose $\bmX$ is a 2D random vector whose components ($X_1$ and $X_2$) can take only two values: 1 or 0, and the four possible combinations of 0 and 1 are equally likely. The support of the discrete vector $\bmX$ is 
	$$R_{\bmX}=\lt\{\begin{pmatrix}
		1 \\ 1
	\end{pmatrix},\begin{pmatrix}
	1 \\ 0
	\end{pmatrix},\begin{pmatrix}
	0 \\ 1
	\end{pmatrix},\begin{pmatrix}
	0 \\ 0
	\end{pmatrix}\rt\}.$$
	
	Its pmf is 
	$$p_{\bmX}=\begin{cases}
		0.25 &\text{ if }x=\begin{pmatrix}
			1 & 1
		\end{pmatrix}^T;\\
		0.25 &\text{ if }x=\begin{pmatrix}
			1 & 0
		\end{pmatrix}^T;\\
		0.25 &\text{ if }x=\begin{pmatrix}
			0 & 1
		\end{pmatrix}^T;\\
		0.25 &\text{ if }x=\begin{pmatrix}
			0 & 0
		\end{pmatrix}^T;\\
		0 & \text{ otherwise.}
	\end{cases}$$
	}
	
	\subsection{Continuous random vectors} % 1.1.2
	\dfn{Continuous random vector}{
	A random vector $\bmX$ is \textbf{continuous} (or \textbf{absolutely continuous}) iif
	\begin{enumerate}
		\item its support $R_{\bmX}$ is uncountable;
		\item there exists a function $f_{\bmX}:\RR^K\rightarrow[0,\infty]$, called the \textbf{joint probability density function} of $\bmX$, such that for any set $A\subseteq\RR^K$ where
		$$A=[a_1,b_1]\times\ldots\times[a_K,b_K].$$
		The probability that $\bmX\in A$ is calculated by
		$$P(\bmX\in A)=\int_{a_1}^{b_1}\ldots\int_{a_K}^{b_K}f_{\bmX}(x_1,\ldots,x_K)dx_K\ldots dx_1$$
		provided the multiple integral is well defined.
	\end{enumerate}
	}
	\ex{}{
	Suppose $\bmX$ is a 2D random vector whose components $X_1$ and $X_2$ are independent uniform random variables on the interval $[0,1]$. Then, $\bmX$ is an example of a continuous vector with support
	$$R_{\bmX}=[0,1]\times[0,1].$$
	Its joint pmf is 
	$$f_{\bmX}(\bmx)=\begin{cases}
		1 & \text{ if }\bmx\in[0,1]\times[0,1];\\
		0 & \text{ otherwise.}
	\end{cases}$$
	The probability that the realisation of $\bmX$ falls in the rectangle $[0,0.5]\times [0,0.5]$ is
	\begin{align*}
		P(\bmX\in[0,0.5]\times [0,0.5])&=\int_{0}^{0.5}\int_{0}^{0.5}f_{\bmX}(x_1,x_2)dx_2dx_1\\
		&=\int_{0}^{0.5}\int_{0}^{0.5}(1)dx_2dx_1\\
		&=\int_{0}^{0.5}[x_2]_0^{0.5}dx_1\\
		&=\int_{0}^{0.5}0.5dx_1\\
		&=[0.5x_1]_0^{0.5}\\
		&=0.25		
	\end{align*}
	}
	\subsection{Random vectors in general} % 1.1.3
	\dfn{Joint distribution function}{
	Let $\bmX$ be a random vector. The \textbf{joint (cumulative) distribution function} of $\bmX$ is a function $F_{\bmX}:\RR^K\rightarrow[0,1]$ such that
	$$F_{\bmX}(\bmx)=P(X_1\leq x_1,\ldots, X_K\leq x_K), \forall \bmx\in\RR^K,$$
	where the components of $\bmX$ and $\bmx$ are denoted by $X_k$ and $x_k$ respectively, for $k=1,\ldots,K$.
	}
	Similarly for the case of joint pmf/pdf, the following notations are used interchangeably to indicate the joint cdf:
	$$F_{\bmX}(\bmx)=F_{\bmX}(x_1,\ldots,x_K)=F_{X_1,\ldots,X_K}(x_1,\ldots,x_K).$$
	
	\subsection{Joint distribution} % 1.1.4
	Sometimes we talk about the \textbf{joint distribution} of a random vector without specifying whether we mean the joint cdf, pmf, or pdf. And this is justified, since the joint pmf/pdf completely determines and is complete determined by the joint cdf of a discrete/continuous vector.
	
	
	\subsection{Random matrices} % 1.1.5
	\dfn{Random matrix}{
	A random matrix is a matrix whose entries are random variables.
	}
	
	A random matrix can always be written as a random vector by vectorising it: given a $K\times L$ random matrix $\bmA$, its vectorisation, denoted $\vect(\bmA)$ is the $KL\times 1$ random vector obtained by stacking the columns of $\bmA$ on top of each other.
	\ex{}{
	Let $\bmA$ be the following $2\times 2$ random matrix:
	$$\bmA=\begin{pmatrix}
		a_{11} & a_{12}\\
		a_{21} & a_{22}
	\end{pmatrix}.$$
	The vectorisation of $\bmA$ is the following $4\times 1$ vector:
	$$\vect(\bmA)=\begin{pmatrix}
		a_{11} \\ a_{21}\\ a_{12} \\ a_{22}
	\end{pmatrix}.$$
	}
	
	If $\vect(\bmA)$ is a discrete/continuous vector, then $\bmA$ is a \textbf{discrete/continuous random matrix}, the joint pmf of $\bmA$ is just the joint pmf/pdf of $\vect(\bmA)$.
	
	\subsection{The marginal distribution of a random vector} % 1.1.6
	Let $X_i$ be the $i$-th component of a $K$-dimensional random vector $\bmX$. The cdf $F_{X_i}(\bmx)$ of $X_i$ is the marginal distribution function of $X_i$.\\
	
	If $\bmX$ is discrete/continuous, then $X_i$ is a discrete/continuous random variable and its pmf $p_{X_i}(\bmx)$/pdf $f_{X_i}(\bmx)$ is the \textbf{marginal pmf/pdf} of $X_i$.
	
	\subsection{Marginalisation of a joint distribution} % 1.1.7
	\textbf{Marginalisation} is the process of deriving the distribution of a component $X_i$ of a random vector $\bmX$ from the joint distribution of $\bmX$.\\
	
	It can also have a broader meaning of deriving the joint distribution of a subset of the set of components of $\bmX$ from the joint distribution of $\bmX$, e.g. if $\bmX$ has three components $X_1,X_2,X_3$, we can marginalise their joint distribution to find the joint distribution of $X_1$ and $X_2$. In this case, $X_3$ is said to be marginalised out of the joint distribution of $X_1,X_2,$ and $X_3$.
		
	\subsection{The marginal distribution of a discrete vector} % 1.1.8
	Let $X_i$ be the $i$-th component of a $K$-dimensional discrete random vector $\bmX$. The marginal pmf of $X_i$ is derived from the joint pmf by:
	
	$$p_{X_i}(x)=\sum_{(x_1,\ldots,x_K)\in R_{\bmX}:x_i=x}p_{\bmX}(x_1,\ldots,x_K),$$
	
	where the sum is over the set
	
	$$\{(x_1,\ldots,x_K)\in R_{\bmX}: x_i=x\},$$
		
	i.e. the probability that $X_i=x$ is obtained as the sum of the probabilites of all the vectors in $R_{\bmX}$ such that their $i$-th component is equal to $x$.
	
	\subsection{Marginalisation of a discrete distribution} % 1.1.9
	Let $X_i$ be the $i$-th component of a $K$-dimensional discrete random vector $\bmX$. By marginalising $\bmX_i$ out of the joint distribution of $\bmX$, we obtain the joint distribution of the remaining components of $\bmX$, $\bmX_{-i}$:
	
	$$\bmX_{-i}=\begin{pmatrix}
		X_1 & \ldots & X_{i-1} & X_{i+1} & \ldots & X_K
	\end{pmatrix}.$$
	
	The joint pmf of $\bmX_{-i}$ is computed as follows:
	
	$$p_{\bmX_{-i}}(x_1,\ldots,x_{i-1},x_{i+1},\ldots,x_K)=\sum_{x_i\in R_{X_i}}p_{\bmX}(x_1,\ldots,x_{i-1},x_{i+1},\ldots, x_K),$$
	
	i.e. the joint pmf of $\bmX_{-i}$ is computed by summing the joint pmf od $\bmX$ over all values of $x_i$ that belong to the support of $X_i$.
	
	\subsection{The marginal distribution of a continuous vector} % 1.1.10
	Let $X_i$ be the $i$-th component of a $K$-dimensional continuous random vector $\bmX$. The \textbf{marginal pdf} of $X_i$ is derived from the joint pdf of $\bmX$ by
	
	$$f_{X_i}(x)=\int_{-\infty}^{\infty}\ldots\int_{-\infty}^{\infty}f_{\bmX}(x_1,\ldots,x_{K})dx_K\ldots dx_{i+1}dx_{i-1} dx_1,$$
	
	i.e. the joint pdf evaluated at $x_i=x$ is integrated with respect to all variables except $x_i$.
	
	\subsection{Marginalisation of a continuous distribution} % 1.1.11
	Let $X_i$ be the $i$-th component of a continuous random vector $\bmX$. By marginalising $X_i$ out of the joint distribution of $\bmX$, we obtain the joint distribution of the remaining components of $\bmX$, $\bmX_{-i}$.\\
	
	The joint pdf of $\bmX_{-i}$ is then computed by
	
	$$f_{\bmX_{-i}}(x_1,\ldots,x_{i-1},x_{i+1},\ldots,x_K)=\int_{-\infty}^{\infty}f_{\bmX}(x_1,\ldots,x_K)dx_i,$$
	
	i.e. the joint pdf of $\bmX_{-i}$ is computed by integrating the joint pdf of $\bmX$ with respect to $x_i$.
	
	\subsection{Partial derivatives of the distribution function of a continuous vector} % 1.1.12
	We know that if $\bmX$ is continuous, then
	
	$$F_{\bmX}(\bmx)=\int_{-\infty}^{x_1}\ldots\int_{-\infty}^{x_K}f_{\bmX}(t_1,\ldots,t_K)dt_K\ldots dt_1.$$
	
	So by taking the $K$-th order cross-partial derivative with respect to $x_1,\ldots,x_K$ of both sides of the above equation, we get
	
	$$\frac{\pt^KF_{\bmX}(\bmx)}{\pt x_1\ldots,\pt x_{K}}=f_{\bmX}(\bmx).$$
	
	\subsection{A more rigorous definition of random vectors} % 1.1.13
	The following is a more rigorous definition of random vector using the formalism of measure theory. (I'll ignore this part for the time being.)
	
	\dfn{}{
	Let $(\Om,\mcF, P)$ be a probability space. Let $\mcB(\Rk)$ be the Borel sigma-algebra of $\Rk$ (i.e. the smallest sigma-algebra containing all open hyper-rectangles in $\Rk$). A function $\bmX:\Om\rightarrow\Rk$ such that 
	$$\{\om\in\Om:\bmX(\om)\in B\}\in\mcF$$
	for any $B\in\mcB(\Rk)$ is said to be a random vector on $\Om$.
	}
	
	This definition ensures that the probability that the realisation of $\bmX$ belongs to a set $B\in\mcB(\Rk)$ that can be defined as
	
	$$P(\bmX\in B)\coloneq P(\{\om\in\Om:\bmX(\om)\in B\})$$
	
	because the set $\{\om\in\Om:\bmX(\om)\in B\}$ belongs to the sigma-algebra $\mcF$ and hence its probability is well-defined.

	
	\subsection{Exercises} % 1.1.14
	\qs{}{
	Let $\bmX$ be a $2\times 1$ discrete random vector and denote its components by $X_1$ and $X_2$.\\
	Let the support of $X$ be the set of all $2\times 1$ vectors such that their entries belong to the set of the first three natural numbers, that is,
	$$R_X=\{\bmx=\begin{pmatrix}
		x_1 & x_2
	\end{pmatrix}^T:x_1\in N_3 \text{ and }x_2\in N_3\}.$$
	}
	
	%%% 1.2 | Expected value %%%
	\section{Expected value}
	In this section we give an informal definition of expected value. A formal definition involves the Lebesgue integral which I will ignore for the time being.\\
	
	\dfn{Expected value}{
	The \textbf{expected value} of a random variable $\bmX$, $E[X]$ is the weighted average of the values that $\bmX$ can take on, where each possible value is weighted by its respective probability.
	}
	
	\subsection{Expected value of a discrete random variable} % 1.2.1
	\dfn{Expected value of a discrete random variable}{
	Let $X$ be a discrete random variable with support $R_{\bmX}$ and pmf $p_{X}(x)$. The expected value of $\bmX$ is
	$$E[X]=\sum_{x\in R_{X}}xp_X(x),$$
	provided that we have \textbf{absolute summability}
	$$\sum_{x\in R_X}|x|p_X(x)<\infty,$$
	ensuring that the summation is well-defined when $\Rx$ contained infinitely many elements.
	}
	
	\nt{
	When summing infinitely many terms, the order in which you sum them can change the results and the expected value of $\bmX$ is not well-defined or does not exist. However this is not true if the terms are absolutely summable. 
	}
	
	\ex{Expected value}{
	Let $\bmX$ be a random variable with support $\Rx=\{0,1\}$ and pmf
	$$p_X(x)=\begin{cases}
		0.5 & \text{ if }x=1;\\
		0.5 & \text{ if }x=0;\\
		0 & \text{ otherwise.}
	\end{cases}$$ 
	Its expected value is
	\begin{align*}
		E[X]&=\sum_{x\in\Rx}xp_X(x)\\
		&=(1)(0.5)+(0)(0.5)\\
		&=0.5
	\end{align*}
	}
	
	
	\subsection{Expected value of a continuous random variable} % 1.2.2
	\dfn{Expected value of a continuous random variable}{
	Let $X$ be a continuous random variable with pdf $f_X(x)$. The expected value of $X$ is 
	$$E[X]=\int_{-\infty}^{\infty}xf_X(x)dx,$$
	provided that we have absolute integrability
	$$\int_{-\infty}^{\infty}|x|f_X(x)dx<\infty.$$
	}
	
	\ex{Expected value of continuous random variable}{
	Let $X$ be a continuous random variable with support $\Rx=[0,\infty)$ and pdf
	$$f_X(x)=\begin{cases}
		\lm\exp(-\lm x) & \text{ if }x\in[0,\infty);\\	
		0 & \text{ otherwise.}
	\end{cases}$$
	where $\lm>0$. Its expected value is 
	\begin{align*}
		E[X]&=\int_{-\infty}^{\infty}xf_X(x)dx\\
		&=\int_{0}^{\infty}x\lm(-\lm x)dx\\
		&=\frac{1}{\lm}\int_{t=0}^{t=\infty}t\exp(-t)dt\\
		&=\frac{1}{\lm}\lt\{[-t\exp(-t)]_{t=0}^{t=\infty}+\int_{0}^{\infty}\exp(-t)dt\rt\}
	\end{align*}
	}
	
	\subsection{Expected value of a random variable in general: the Riemann-Stieltjes integral} % 1.2.3
	\dfn{Expected value (general)}{
	Let $X$ be a random variable with cdf $F_X(x)$. The expected value of $X$ is
	$$E[X]=\int_{-\infty}^{\infty}xdF_X(x),$$
	where the integral is a Riemann-Stieltjes integral and the expected value exists and is well-defined iif the integral is well-defined.
	}
	
	This definition gives a formal notation which allows for a unified treatment of discrete and continuous random variables and can be treated as a sum in one case and as ordinary Riemann integral in the other.
	
	\ex{}{
	Let $X$ be a random variable with support $R_X=[0,1]$ and distribution function
	$$F_X(x)=\begin{cases}
		0 & \text{ if }x<0\\
		0.5x & \text{ if }0\leq x<1\\
		1 & \text{ if }x\geq 0
	\end{cases}.$$
	Its expected value is
	\begin{align*}
		E[X]&=\int_{-\infty}^{\infty}xdF_X(x)\\
		&=\int_{0}^{1}xdF_X(x)+1\cdot\lt[F_X(1)-\lim_{x\rightarrow 1^+}F_X(x)\rt]\\
		&=\int_{0}^{1}x\frac{d}{dx}(\frac{1}{2}x)dx+1\cdot[1-\frac{1}{2}]\\
		&=\lt[\frac{1}{4}x^2\rt]_0^1+\frac{1}{2}\\
		&=\frac{3}{4}
	\end{align*}
	}
	
	\subsection{Expected value of a random variable in general: the Lebesgue integral} % 1.2.4
	\dfn{Expected value (rigorous)}{
	Let $\Om$ be a sample space, $P$ a probability measure defined on the events of $\Om$ and $X$ a random variable defined on $\Om$. The expected value of $X$ is 
	$$E[X]=\int XdP,$$
	provided the Lebesgue integral of $X$ with respect to $P$ exists and is well-defined.
	}
	
	\subsection{The transformation theorem} % 1.2.5
	Let $X$ be a random variable, $g:\RR\rightarrow\RR$ be a real function. Define a new random variable $Y$ as $Y=g(X)$. Then, 
	
	$$E[Y]=\int_{-\infty}^{\infty}g(x)dF_X(x)$$
	
	provided that the integral exists. For discrete random variables, the formula becomes
	
	$$E[Y]=\sum_{x\in\Rx} g(x)p_X(x)$$
	
	while for continuous random variables, 
	
	$$E[Y]=\int_{-\infty}^{\infty}g(x)f_X(x)dx.$$
	
	When $\bmX$ is a discrete random vector and $p_{\bmX}(x)$ is its joint pmf, then
	
	$$E[Y]=\sum_{\bmx\in\RX}g(\bmx)p_{\bmX}(\bmx).$$
	
	When $\bmX$ is an continuous random vector and $f_{\bmX}(\bmx)$ is its joint pdf, then
	
	$$E[Y]=\int_{-\infty}^{\infty}\ldots\int_{-\infty}^{\infty}g(\bmx)f_{\bmX}(\bmx)dx_1\ldots dx_K.$$
	
	\subsection{Linearity of the expected value} % 1.2.6
	\thm{}{If $X$ is a random variable and $Y$ is another random variable such that
		$$Y=a+bX,$$
		where $a,b\in\RR$, then we have
		$$E[Y]=a+bE[X].$$}
	\begin{myproof}
		For discrete random variables,
		\begin{align*}
			E[Y]=\sum_{x\in\Rx}(a+bx)p_X(x)
		\end{align*}
	\end{myproof}
	
	\subsection{Expected value of random vectors} % 1.2.7
	\subsection{Expected value of random matrices} % 1.2.8
	\subsection{Integrability and Lp spaces} % 1.2.9
	\subsection{Exercises} % 1.2.10
	
	%%% 1.3 | Properties of expected value %%%
	\section{Properties of the expected value}
	\subsection{Scalar multiplication of a random variable} % 1.3.1
	\subsection{Sums of random variables} % 1.3.2
	\subsection{Linear combination of random variables} % 1.3.3
	\subsection{Expected value of a constant} % 1.3.4
	\subsection{Expectation of a product of random variables} % 1.3.5
	\subsection{Non-linear transformations} % 1.3.6
	\subsection{Addition of a constant matrix and a matrix with random entries} % 1.3.7
	\subsection{Multiplication of a constant matrix and a matrix with random entries} % 1.3.8
	\subsection{Expectation of a positive random variable} % 1.3.9
	\subsection{Preservation of almost sure inequalities} % 1.3.10
	\subsection{Exercises} % 1.3.11
	
	%%% 1.4 | Variance %%%
	\section{Variance}
	\section{Covariance}
	\section{Linear correlation}
	\section{Covariance matrix}
	\section{Indicator functions}
	\section{Quantile}
	
	\chapter{Conditional distributions and independence}
	\subsection{Rigorous conditional probebility}
	
\end{document}